

\section*{Basics of amplitudes}

The 2-to-2 scattering of identical massive ($m=1$) scalar particles is encoded by the $S$-matrix. Above threshold $s > 4$ and in any dimension $d\geq2+1$, it satisfies \textbf{unitarity}
\begin{equation}
\hat S^\dagger \hat  S=\mathbb{1}\,, \qquad
S=1+i (2\pi)^d \delta^{(d)}(p_1+p_2-p_3-p_4) M(s,t,u)\,,
\label{eq:unitarity} 
\end{equation}
with $S=\bra{p_3,p_4} \hat{S} \ket{p_1, p_2}$ and the onshell momenta $p_i$ are expressed with Mandelstam invariants
\begin{equation}
s=(p_1+p_2)^2\,,\quad
t=(p_1-p_3)^2\,,\quad
u=(p_1-p_4)^2\,,\quad
s+t+u=4\,.
\end{equation}
We look at the amplitude for external, identical, massive scalars (lightest particles in the spectrum). The amplitude $M(s,t)\equiv M(s,t,u=4-s-t)$ satisfies \textbf{crossing} symmetry
\begin{equation}
M(s,t)=M(t,s)=M(u,t)\,,
\end{equation}
as well as real \textbf{analyticity} 
\begin{equation}
M(s^*,t^*)=M^*(s,t)\,.
\end{equation}
In the set-up below, we further assume {\it maximal} analyticity, under its form of lightest particle maximal analtycity.




\section*{Partial Wave decomposition}
We consider the partial wave expansion in arbitrary dimension $d\geq 2+1$
\begin{equation}
M(s,t)=\sum_{\ell=0}^{\ell_\mathrm{max}}n_{d,\ell}\,P_{d,\ell}(x)\,f_\ell(s)\,,
\qquad
P_{d,\ell}(x)={}_2F_1\left(-\ell,d+\ell-3,\frac{d-2}{2},\frac{1-x}{2}\right)\,,
\end{equation}
with (even) spin $\ell$ (in principle $\ell_\mathrm{max}\rightarrow\infty$) and conventional normalization factor $n_{d,\ell}$. Inversing using orthogonality of the partial waves $P_{d,\ell}(x)$, gives the partial wave coefficients
\begin{equation}
f_\ell(s)=\frac{\mathcal{N}_d}{2}\int_{-1}^{1}\mathrm{d}x\,(1-x^2)^{\frac{d-4}{2}}\,P_{d,\ell}(x)\,M(s,t(s,x))\,, 
\qquad
x=\cos\theta=1+\frac{2t}{s-4}\,,
\end{equation}
with $\theta$ the scattering angle and $\mathcal{N}_d$ another conventional normalization factor.
The unitarity condition \eqref{eq:unitarity} becomes
\begin{equation}
|S_\ell(s)|^2\leq 1\,,
\qquad
S_\ell(s)=1+\mathrm{i}\,\phi^2(s)\,f_\ell(s)\,, 
\qquad
\phi^2(s)=\frac{(s-4)^{\frac{d-3}{2}}}{\sqrt{s}}\,.
\label{eq:unitarity2}
\end{equation}
Our normalization conventions read
\begin{equation}
n_{d,\ell}(x)=\frac{(4\pi)^{d/2}(d+2\ell-3)\Gamma(d+\ell-3)}{\pi\,\Gamma\left(\frac{d-2}{2}\right)\Gamma(\ell+1)}\,,
\qquad
\mathcal{N}_d=\frac{(16\pi)^{\frac{d-2}{2}}}{\Gamma\left(\frac{d-2}{2}\right)}\,.
\end{equation}



\section*{Wavelet ansatz}
We consider the amplitude ansatz
\begin{equation}
M(s,t)=\sum_{\alpha\beta\gamma\delta}\,\alpha_{\alpha\beta\gamma\delta}\delta_{\alpha\beta\gamma\delta}\,R_{\alpha\beta\gamma\delta}(s,t)\,,
\end{equation}
with $\alpha_{\alpha\beta\gamma\delta}$ our free real parameters, the $\rho$-wavelet ansatz
\begin{equation}
R_{\alpha\beta\gamma\delta}(s,t)=\frac{n_d}{6}\,\left(\rho^\gamma(s,\sigma_\alpha)\rho^\delta(t,\sigma_\beta)+ s \leftrightarrow t \leftrightarrow u \right)\,, 
\end{equation}
with $n_d=2^{5-d}n_{d,0}$ and the conformal map 
\begin{equation}
\rho(s,\sigma)=\frac{\sqrt{\sigma-4}-\sqrt{4-s}}{\sqrt{\sigma-4}+\sqrt{4-s}}\,,
\label{eq:confmap}
\end{equation}
with the wavelet centers indices $\alpha,\beta\in \{1,2,...,N_{\mathrm{max}}\}$ with $\sigma_n$ some grid \eqref{eq:sigmagrid} and $\rho$ powers $\alpha,\beta\in \pi_n=\{0,1,...\}$, together with the tricky boolean condition to avoid double counting:
\begin{equation}
\delta_{\alpha\beta\gamma\delta}=\delta\big(
\sigma_\alpha\leq\sigma_\beta \land 
\gamma\leq \delta \land 
(\gamma\neq0 \lor \alpha\leq 1) \land 
(\delta\neq0 \lor \beta\leq 1) \land
\gamma\leq \min(\pi_\alpha,\pi_\beta) \land
\delta\leq \max(\pi_\alpha,\pi_\beta)
\big)\,.
\end{equation}




\section*{Grid and numerics}
Our grids are based on the Chebychev grid
\begin{equation}
\phi_n=\frac{\pi}{2}\left(1+\cos\frac{n\pi}{N+1}\right)\,, \qquad n=1,...,N\,, \qquad \phi_n\in (0,\pi)\,.
\end{equation}
The $s$ grid (dense near threshold, then more and more sparse) reads
\begin{equation}
s_n=s(e^{i\phi_n},\sigma_0)\,, \qquad n=1,...,s_\mathrm{max}\,, \qquad s_n \in (4,\infty)\,,
\end{equation}
where $s(\rho,\sigma)$ is obtained by inverting the conformal map \eqref{eq:confmap} and $\sigma_0$ stands for the wavelet center mother point, obtained by solving $\rho(s_0,\sigma_0)=0$ with $s_0$ the crossing symmetric point $s_0=t_0=u_0=4/3$, leading to $\sigma_0=20/3$. The wavelet centers cumulative grid reads
\begin{equation}
\sigma_n=\sigma_{n-1}\cup s_{i,j}\,,
\qquad n=1,...,N_\mathrm{max}\,, \qquad 
\sigma_n \in (4,\infty)\,.
\label{eq:sigmagrid}
\end{equation}
where $\sigma_1=s_1$ and $(i,j)$ is the smallest pair such that $s_{i,j}\notin \sigma_{n-1}$.
The (even) spin grid is simply 
\begin{equation}
\ell=\{0,2,...,\ell_\mathrm{max}\}\,,
\end{equation}
and the $\rho$ ansatz power grid is trivially
\begin{equation}
\pi_n=n\,, \qquad n=0,...,\pi_{\mathrm{max}}\,.
\end{equation}
The numerics achieve perfection at $\ell_\mathrm{max}\rightarrow\infty$, $s_\mathrm{max}\rightarrow\infty$, $N_\mathrm{max}\rightarrow\infty$ and $\pi_{\mathrm{max}}\geq1$ (one could also choose instead $N_\mathrm{max}\geq1$ and $\pi_{\mathrm{max}}\rightarrow\infty$). For reasonable convergence, we typically use $\pi_\mathrm{max}=1$, $\ell_\mathrm{max}\approx14$ spins or more, $s_\mathrm{max}=300$ points, so $s$ range from $4+10^{-9}$ up to $10^9$ and $N_\mathrm{max}=1$ to $22$ wavelet centers, yielding for example $\sigma_{22,i}=\{4.006, 4.010, 4.016, ... , 4.970, 5.306, 6.667, 9.444, 11.33, ... , 243.1, 443.0, 748.3\}$, so we expect our amplitude to satisfy elastic unitary roughly up to $s\approx100$. For reference, the wavelets are shot in this order: 
\begin{equation}
\begin{array}{|c|c|c|c|c|c|c|c|c|c|c|}
\hline
N_{\mathrm{max}} & 1 & 2 & 3 & 4 & 5 & 6 & 7 & 8 & 9 & 10 & 11 \\ \hline
\sigma & 6.667 & 4.458 & 19.54 & 4.146 & 52.62 & 4.061 & 4.970 & 11.33 & 120.7 & 4.030 & 243.1 \\ \hline
\end{array}
\end{equation}
\begin{equation}
\begin{array}{|c|c|c|c|c|c|c|c|c|c|c|}
\hline
N_{\mathrm{max}} & 12 & 13 & 14 & 15 & 16 & 17 & 18 & 19 & 20 & 21 & 22 \\ \hline
\sigma & 4.016 & 4.247 & 5.306 & 9.444 & 32.73 & 443.0 & 4.010 & 4.740 & 13.61 & 748.3 & 4.006 \\ \hline
\end{array}
\end{equation}
At typical $N_{\mathrm{max}}=22$, the ansatz contains 277 $\alpha$ parameters for a typical total number of non-linear constraints of $s_{\mathrm{max}}\times(1+\ell_{\mathrm{max}}/2)=2700$ at typical values $s_{\mathrm{max}}=300$ and $\ell_{\mathrm{max}}=16$.




\section*{Wilson coefficients}
For a weakly coupled amplitude, the non-perturbative couplings $c_n$ can be related to the expansion of the amplitude around the crossing symmetry point (cs:  $s_0=t_0=u_0=4/3$), given by
\begin{equation}
M(s,t,u)=
\frac{2}{\mathcal{N}_d}\left(
c_0
+\frac{c_2}{2}\left(\bar{s}^2+\bar{t}^2+\bar{u}^2\right)
+\frac{c_3}{2}\left(\bar{s}\bar{t}\bar{u}\right)
+O\left(\bar{s}^4,\bar{t}^4,\bar{u}^4\right)
\right)\,,
\end{equation}
where $\bar{x}=x-4/3$. The $c_n$ are then defined as derivatives of the amplitude taken at the crossing symmetric point 
\begin{equation}
c_n=\frac{\mathcal{N}_d}{2}\times 
\begin{cases}
c(0,0) & n=0 \\
\frac{1}{2}c(n,0) & n \text{ even} \\
c(n-1,1) & n>1 \text{ odd}
\end{cases}\,\,\,,
\qquad
c(n_1,n_2)=\left.\frac{1}{n_1!n_2!}\partial_s^{n_1}\partial_t^{n_2}M(s,t)\right|_\mathrm{cs}\,,
\end{equation}
These have a nice interpretation in perturbative field theory. Let's consider the weakly coupled EFT of a single massive ($m=1$) field $\phi$ with Lagrangian density
\begin{equation}
\mathcal{L}=
\frac{1}{2}(\partial_\mu\phi)^2
-\frac{1}{2}\phi^2
-\frac{1}{4!}g_0\phi^4
+\frac{1}{2}\frac{g_2}{\Lambda^4}(\partial_\mu\phi)^4
+\frac{1}{3}\frac{g_3}{\Lambda^6}(\partial^\mu\partial_\rho\phi)(\partial^\nu\partial_\mu\phi)(\partial^\rho\partial_\nu\phi)\phi
+ O\left(\phi^6/\Lambda^8\right)
\label{eq:lagrangian}
\end{equation}
with the cutoff scale $[\Lambda]=1$ and $[g_n]=4-d$. It is then straightforward to compute the amplitude $M(s,t)$, reading
\begin{equation}
M(s,t)=-g_0
+\frac{g_2}{\Lambda^4}\left(s^2+t^2+u^2-4\right)
+\frac{g_3}{\Lambda^6}(s-2)(t-2)(u-2)
+O\left(1/\Lambda^8\right)\,.
\end{equation}
This leads to the following mapping between the $c_n$ and $g_n$
\begin{equation}
\begin{aligned}
c_0 & = \frac{\mathcal{N}_d}{2}\left(-g_0+\frac{4}{3}\frac{g_2}{\Lambda^4}-\frac{8}{27}\frac{g_3}{\Lambda^6}+O\left(1/\Lambda^8\right)\right)\,, \\
c_2 & = \frac{\mathcal{N}_d}{2}\left(+\frac{g_2}{\Lambda^4}+\frac{1}{3}\frac{g_3}{\Lambda^6}+O\left(1/\Lambda^8\right)\right) \,, \\
c_3 & = \frac{\mathcal{N}_d}{2}\left(-\frac{g_3}{\Lambda^6}+O\left(1/\Lambda^8\right)\right) \,.
\end{aligned}
\end{equation}



\section*{Optimization problem}
Unitary \eqref{eq:unitarity2} can be written with the matrix
\begin{equation}
\det U_\ell(s)\geq1\,, \qquad
U_\ell(s)=\begin{pmatrix}
1-\frac{1}{2}\phi^2(s)\operatorname{Im} f_\ell(s) &
\phi(s)\operatorname{Re}f_\ell(s)  \\
\phi(s)\operatorname{Re}f_\ell(s) &
2\operatorname{Im}f_\ell(s)
\end{pmatrix}\,,
\end{equation}
and the game is then to optimize (maximize/minimize) a given $c_n$ using the semi-definite program 
\begin{equation}
\begin{aligned}
\text{optimize} \quad & c_n \\
\text{over} \quad & \alpha_{\alpha\beta\gamma\delta} \\
\text{subject to} \quad & U_\ell(s)\succeq 0 \quad \text{sampled over }\ell, s
\end{aligned}\,.
\end{equation}
This gives us $c_{n\,\mathrm{min/max}}$ (typically $n=0,2,3$), and fixes all the corresponding $\alpha_{\alpha\beta\gamma\delta}$ parameters, hence the $S$-matrix $S_\ell(s)$.




\section*{Convergence improvements}
We may add Subtracted Positivity Constraints (SPC) 
\begin{equation}
\operatorname{Im}\left(M(s,t)-
\sum_\ell^{\ell_\mathrm{max}} n_{d,\ell}
P_{d,\ell}\left(1+\frac{2t}{s-4}\right)
f_\ell(s)
\right)\geq0\,,
\end{equation}
sampled over $s_i$ and also over $t_i$ for a few points (typically $t_{\mathrm{max}}=10$ points, i.e., 5 from 0 to 4, and 5 very close to $4^-$). This generate an additional $t_{\mathrm{max}}\times s_{\mathrm{max}}\times (1+\ell_{\mathrm{max}}/2) =27000$ linear constraints to the optimization problem at typical values $s_{\mathrm{max}}=300$ and $\ell_{\mathrm{max}}=16$. These constraints greatly improve the convergence in $\ell$ so that $\ell_{\mathrm{max}}=16$ achieve near perfect convergence already. 




\section*{Threshold term}
We also may add the Threshold (TH) improvement term
\begin{equation}
M(s,t)\rightarrow M(s,t)+\alpha_{\mathrm{th}}\Delta(s,t)\,,
\end{equation}
with
$\Delta(s,t)=\Delta(s)+\Delta(t)+\Delta(u)$, and 
\begin{equation}
\Delta(s)=
n_d\times
\left(\cfrac{4-s}{4}\right)^{-\frac{d-3}{2}}\times
\begin{cases}
0 & d=3 \\
\cfrac{(-1)^{\frac{d-3}{2}}\pi}{\log{\frac{4-s}{4}}+1} & d \text{ odd} \\
\sin\left(\pi\frac{d-3}{2}\right) & \text{else}
\end{cases}\,\,\,,
\end{equation}
which is normalized so that $\alpha_{\mathrm{th}}\in[0,1]$.


